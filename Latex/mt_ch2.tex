\chapter{Examples and Non-examples}

In this chapter, we will explore the Einstein relation by examining some examples and will thusly motivate some of the more general results of the next chapter.

\section{Euclidean Space}

\section{Sierpinski Gasket}

The Sierpinski Gasket is a simple example of an iterated function fractal and can be described according to theorem \ref{thm:hutchinson} as the unique non-empty compact set $\SG\ssq\IR^2$ which is invariant under the three similitudes 
\[
  S_1(x,y)=\left(\frac{x}{2},\frac{y}{2}\right),\ 
  S_2(x,y)=\left(\frac{x+1}{2},\frac{y}{2}\right),\ 
  S_3(x,y)=\left(\frac{2x+1}{4},\frac{2y+\sqrt{3}}{4}\right),
\]
see PICTURE! Since $\SG$ satisfies the (OSC), e.g. by taking the open equilateral triangle with corners $(0,0), (0,1)$ and $(1/2,\sqrt{3}/2)$, we obtain both 
\begin{equation}\label{eq:SGdimh}
  s=\dimh(\SG)=\frac{\ln 3}{\ln 2}
\end{equation}
and $\cH^s(\SG)\in(0,\infty)$ by a second appeal to Hutchinson's theorem. 


\section{Combs and inhomogenous graphs}

In this section, we start by considering the graph $\bC_2$, called the two-dimensional integer comb, with vertex set $\IZ^2$ and edge set given by
\[
  \left\{\{(n_1,n_2),(m_1,m_2)\}\in\IZ^2\times\IZ^2:|n_1-m_1|=1,n_2=m_2=0\ \text{ or }\ |n_2-m_2|=1,n_1=m_1\right\}
\]


\section{Bounded metric spaces}