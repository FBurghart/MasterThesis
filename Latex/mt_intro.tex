\chapter*{Introduction}
\addcontentsline{toc}{chapter}{Introduction}
\markboth{INTRODUCTION}{Introduction}

When regarding an open bounded domain $U$ in $\IR^n$, the Einstein relation is an equation expressing that the geometric behavior -- expressed in the asymptotic scaling of mass for small balls -- is nicely compatible with the analytic structure given by the Dirichlet-Laplace operator $\gD$ on $U$ -- expressed in the asymptotic behaviour of its eigenvalue counting function -- and with the asymptotic velocity of the stochastic process induced by $\gD$, namely Brownian motion. 
With the development of analytic and stochastic theory on (mainly self-similar) fractals, it was also discoverd that the same relation holds on some fractals, most prominently on the Sierpinski gasket $\SG$. 

The main goal of this thesis is to provide a general framework for the Einstein relation. To achieve this, we consider metric measure spaces $(X,d_X,\mu_X)$, where $(X,d_X)$ is a complete, separable, locally compact, and path-connected metric measure space (not consisting of only a single point) with an everywhere supported Radon measure $\mu_X$ on it. 

The purpose of the first chapter is to formally introduce the Hausdorff dimension $\dimh$, the spectral dimension $\dims$ and the walk dimension $\dimw$. For $\dimh$, we give a short sketch of its definition and some of its properties, including Hutchinson's theorem on the Hausdorff dimension of self-similar sets in $\IR^n$. The walk dimension is first motivated by Weyl's result on the asymptotic growth of the Dirichlet-Laplacian's eigenvalue counting function and then defined for operators $A$ on $L^2(X,\mu_X)$ that satisfy certain conditions. These conditions also ensure that there exists an essentially unique Hunt process having $A$ as its inifinitesimal generator. The outline of this theory relating processes, operator semigroups, generators and Dirichlet forms is then given in section 1.2 and 1.3 before the (local) walk dimension and then the Einstein relation are defined.

In the second chapter, we begin by examining the above mentioned classical case of a domain in Euclidean space with Dirichlet-Laplace operator, and continue by presenting the constructions of the standard Laplace operator on the Sierpinski gasket as well as the construction of the Brownian motion on $\SG$. Both constructions rely heavily on the self-similarity and on the fact that $\SG$ can be approximated by a sequence of graphs. This also requires a different approach to the walk dimension than the local one from section 1.3, as the vertex set of a graph is always discrete. To see why this graph-theoretic walk-dimension can not be directly adapted to metric measure spaces, we present a counterexample in section 2.3.

The third chapter begins by defining two different types of morphisms between mm-spaces, namely contractions and Lipschitz-maps, both of which give rise to a notion of isomorphy, mm-isomorphism and Lipschitz-isomorphisms, respectively. We then continue to investigate how we can transport the structure needed for the Einstein relation alongside maps $\gp:(X,d_X,\mu_X)\to (Y,d_Y,\mu_Y)$ and prove that the Einstein relation is invariant under Lipschitz-isomorphisms. We proceed by looking at H\"older continuous transformations and manage to proof upper bounds for the walk dimension and apply this to Brownian motions running on the graphs of independent fractional Brownian motion, which generates a family of examples where the Einstein relation might hold with a constant factor different from 1.  

The concluding discussion contains several open questions that aim to further a general theory of the Einstein relation as an invariant of metric measure spaces.

Finally, I would like to thank Uta Freiberg for introducing me to this topic and for supervising this work; her group of PhD students for various forms of help; Frank Calisse for pointing out literature; Matthias Beck for providing me some of his \TeX-Code which I used to generate the graphics in chapter 2; as well as everyone who provided direct or indirect support in any form. 
