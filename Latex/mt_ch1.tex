\chapter{Fractal Dimensions and the Einstein Relation}

In this introductory chapter, we wish to briefly expose the ingredients of the ER - the Hausdorff dimension $\dimh$, the spectral dimension $\dims$, and the walk dimension $\dimw$ - and state some of their properties. 

\section{Hausdorff measure and Hausdorff dimension}

Although the concepts of Hausdorff measure and dimension are well-known, we give the definitions in the interest of completeness. In what follows, let $(E,d)$ be a metric space.
\begin{defin}[Hausdorff outer measure]
  For fixed $s\geq0$, any subset $S\ssq E$ and any $\gd>0$, let 
  \[
    \cH^s_\gd(S)
      :=\inf \left\{\sum_{i\in I}(\diam U_i)^s:
            |I|\leq\aleph_0,S\ssq\bigcup_{i\in I}U_i\ssq E,\diam U_i\leq\gd\right\},
  \]
  i.e. the infimum is taken over all countable coverings of $S$ with diameter at most $\gd$. The $s$-dimensional Hausdorff outer measure of $S$ is now defined to be
  \begin{equation}\label{eq:DHM}
    \cH^s(S):=\lim_{\gd\searrow0}\cH^s_\gd(S).
  \end{equation}
\end{defin}
Observe that the limit in \eqref{eq:DHM} exists or equals $\infty$, since $\cH^s_\gd(S)$ is monotonically nonincreasing in $\gd$, yet bounded from below by 0. Furthermore, it can be shown that $\cH^s$ defines a metric outer measure on $E$, and thus restricts to a measure on a $\gs$-algebra containing the Borel $\gs$-algera $\scB(E)$ (cf. \cite[p.54ff]{mattila1999geometry}). By definition, the obtained measure then is the $s$-dimensional Hausdorff measure which we will denote by $\cH^s$ as well. Note that for $\cH^s$ to be a Radon measure, i.e. locally finite and inner regular, $\cH^s(E)<\infty$ is sufficient.

In the special case of $(E,d)$ being an Euclidean space, Hausdorff measures interpolate between the usual Lebesgue measures $\gl^n$: For $s=0$, we have simply $\cH^0(S)=\# S$, whereas for any integer $n>0$, it can be shown that there exists a constant $c_n>0$ depending only on $n$ such that $\cH^n=c_n\gl^n$, where the constant evaluates to the volume of the unit ball.

Since exponential functions are monotonically increasing, the Hausdorff measures' dependence on $s$ exhibits the same behaviour for a fixed set $S$. At the same time, simple estimates yield that if $\cH^s(S)$ is finite for some $s$, it vanishes for all $s'<s$, and conversely, if $\cH^s(S)>0$, then $\cH^{s'}(S)=\infty$ for all $s'>s$. Therefore, there exists precisely one real number $s$ where $\cH^\cdot(S)$ jumps from $0$ to $\infty$ (by possibly attaining any value of $[0,\infty]$ there). This motivates the following definition of Hausdorff dimension:
\begin{defin}
  The Hausdorff dimension $\dimh(S)$ of $S\ssq E$ is defined as
  \[
    \dimh(S):=\inf\{s\geq0:\cH^s(S)>0\}.
  \]
\end{defin}
Due to the above discussion, we have the following equalities:
\begin{align*}
   \dimh(S)&=\inf\{s\geq0:\cH^s(S)>0\}=\inf\{s\geq0:\cH^s(S)=\infty\}\\
           &=\sup\{s\geq0:\cH^s(S)=0\}=\sup\{s\geq0:\cH^s(S)<\infty\},
\end{align*}
providing some alternative characterisations of the Hausdorff dimension.

We further collect some important facts. To this end, let $S,S'$ and $S_1,S_2,...$ be subsets of $E$ as before. Then, the following properties hold (cf. \cite[p.32f]{falconer2007fractal} for a discussion in the Euclidean setting; however all arguments adapt to our more general situation without complication):
\begin{description}
  \item[Monotonicity.] If $S\ssq S'$ then $\dimh(S)\leq\dimh(S')$. 
  \item[Countable Stability.] For a sequence $(S_n)_{n\geq1}$, we have the equality
  \[ 
    \dimh\left(\bigcup_{n\geq1}S_n\right)=\sup_{n\geq1} \dimh(S_n).
  \]
  \item[Countable Sets.] If $|S|\leq\aleph_0$ then $\dimh(S)=0$.
  \item[H\"older continuous maps.] If $(E',d')$ is another metric space and $f:E\to E'$ is $\ga$-H\"older continuous for some $\ga\in(0,1]$ then $\dimh(f(S))\leq \ga^{-1}\dimh(S)$. In particular, the Hausdorff dimension is invariant under a bi-Lipschitz transformation (i.e. an invertible map $f$ with H\"older exponent $\ga=1$ for both $f$ and $f^{-1}$).
  \item[Euclidean Case.] If $(E,d)$ happens to be an Euclidean space (or more generally a continuously differentiable manifold) of dimension $n$ and $S$ is an open subset then $\dimh(S)=n$.
\end{description}

We conclude this section by citing Hutchinson's theorem about the Hausdorff dimension of self-similar sets which will provide us with a plethora of interesting examples. For this, we recall that a map $F:E\to E$ on a metric space $(E,d)$ is a strict contraction if its Lipschitz constant satisfies
\[
  \Lip_F:=\sup_{\stackrel{x,y\in E}{x\neq y}}\frac{d(F(x),F(y))}{d(x,y)}<1.
\]
If the stronger condition $d(F(x),F(y))=\Lip_F d(x,y)$ holds for all $x,y\in E$, we call $F$ a similitude with contraction factor $\Lip_F$.
\begin{thm}[Hutchinson, \cite{hutchinson1981fractals}]\label{thm:hutchinson}
  Let $\cS=\{S_1,...,S_N\}$ be a finite set of strict contractions on the Euclidean space $\IR^n$. Then there exists a unique nonempty compact set denoted by $|\cS|$ invariant under $\cS$, i.e.
  \[
    |\cS|=\bigcup_{i=1}^N S_i(|\cS|).
  \]
  Furthermore, assume that $|\cS|$ satisfies the open set condition (OSC) which means that there exists a nonempty open set $O\ssq E$ with the properties $S_i(O)\ssq O$ and $S_i(O)\cap S_j(O)=\emptyset$ for all $i,j=1,...,N$ with $i\neq j$. Also assume that the maps $S_i$ are similitudes with contraction factor $r_i\in(0,1)$. Then, $s=\dimh(|\cS|)$ is the unique solution to the equation
  \[
    \sum_{i=1}^N r_i^s=1
  \]
  and we have $0<\cH^s(|\cS|)<\infty$. 
\end{thm}
While uniqueness and existence of $|\cS|$ are still ensured for maps on a complete metric space, the open set condition is not sufficient for statements about the Hausdorff dimension, see \cite{schief1996self} for further discussion. 



\section{Weyl asymptotics and spectral dimension}

The idea of introducing spectral dimension is inspired by Weyl's law for the eigenvalues of the Dirichlet-Laplace operator which we will discuss here shortly:

Given a bounded open domain $E\ssq\IR^n$, consider the Laplace operator $\gD$ on $E$ acting on functions that satisfy the Dirichlet boundary condition $u\equiv0$ on $\partial E$. Then, the spectrum of $-\gD$ consists of non-negative eigenvalues with a single limit point at $\infty$. Hence we can order them in a non-increasing way, counting the geometric multiplicities, as
\[
  0\leq\gl_1\leq\gl_2\leq...\leq\gl_n\leq...\ \text{ with }\ \gl_n\nearrow\infty. 
\]
In this setting, it makes sense to define the eigenvalue counting function via 
\begin{equation}\label{eq:DECF}
  N_{-\gD}(x):=\max\{n\in\IN:\gl_n\leq x\}.
\end{equation}
Weyl's law now states that there is the asymptotic equivalence\footnote{We adopt the notation $f\sim g$ for the equivalence relation given by $\lim\frac{f}{g}=1$.}
\begin{equation}\label{eq:WL}
  N_{-\gD}(x)\sim C(n,E)x^{n/2},\quad x\nearrow\infty,
\end{equation}
where the constant $C(n,E)$ depends only on $E$ and its dimension (see \cite{Weyl1911} and \cite{Weyl1912} for the original publications).



\newpage

\section{Diffusion processes and walk dimension}

\subsection{From Dirichlet forms to Markov processes}

We start with the following definition (cf. \cite[Def. IV.1.13]{ma2012introduction}):
\begin{defin}
  Given a filtered probability space $\left(\gO,\scA,\scF=\left(\scF_t\right)_{t\geq0},\Prob\right)$ satisfying the usual conditions, an $\scF$-adapted time-homogenous Markov process $X=(X_t)_{t\geq0}$ with state space $E_\gD$ is called a right process if it satisfies the strong Markov property for all $\scF$-stopping times and all its trajectories are right continuous.  
\end{defin}

\section{Other versions of the Einstein relation}

Telcs, Mandelbrot, etc...

