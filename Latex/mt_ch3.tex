\chapter{The Einstein Relation on Metric Measure Spaces}

This chapter is devoted to the investigation of the Einstein relation in the setting of an abstract mm-space. First, we focus on the behaviour under morphisms between mm-spaces to derive invariance properties of the Einstein relation. 

\section{The Einstein Relation under Lipschitz-isomorphisms}

\subsection{Lipschitz and mm-isomorphisms}

We will use this section to introduce two different categories $\mathsf{MM}_L$ and $\mathsf{MM}_{\leq1}$ whose objects are mm-spaces, but with different morphisms: 
\begin{itemize}
  \item In $\mathsf{MM}_L$, the set $\mathsf{MM}_L(X,Y)$ of morphisms from an object $X=(X,d_X,\mu_X)$ to another object $Y=(Y,d_Y,\mu_Y)$ is the set of all Lipschitz-continuous functions 
  \[ 
    \gp:\supp\mu_X\to\supp\mu_Y
  \]
  satisfying $\gp_*\mu_X=\mu_Y$.
  \item In $\mathsf{MM}_{\leq1}$, the set $\mathsf{MM}_{\leq1}(X,Y)$ of morphisms from an object $X=(X,d_X,\mu_X)$ to another object $Y=(Y,d_Y,\mu_Y)$ is the subset of $\mathsf{MM}_L(X,Y)$ consisting of all contraction maps (i.e. Lipschitz-continuous functions $f$ with $\Lip_f\leq1$, cf. \eqref{eq:Lip}).
\end{itemize}
In both of those categories, composition of morphisms is to be understood as the usual composition of maps. By definition, $\mathsf{MM}_{\leq1}$ is a subcategory of $\mathsf{MM}_L$. Considering the usual notion of isomorphism, both categories give rise to a meaningful concept of isomorphy for mm-spaces: 
\begin{defin}
  A Lipschitz-isomorphism between mm-spaces $(X,d_X,\mu_X)$ and $(Y,d_Y,\mu_Y)$ is a map 
  $\gp:\supp\mu_X\to\supp\mu_Y$ with $\gp_*\mu_X=\mu_Y$ satisfying the bi-Lipschitz condition
  \[
    \frac{1}{C}d_X(x,y)\leq d_Y(\gp(x),\gp(y))\leq Cd_X(x,y)
  \]
  for all $x,y\in\supp\mu_X$ and a constant $C\in[1,\infty)$ not depending on $x,y$.
  
  Similarly, an mm-isomorphism is defined to be a Lipschitz-isomorphism with constant $C=1$. (This coincides with definition 2.8 in \cite{shioya2016metric})
\end{defin}
As it turns out, Lipschitz-isomorphisms are precisely the isomorphisms in $\mathsf{MM}_L$, whereas mm-isomorphisms are the ones in $\mathsf{MM}_{\leq1}$.

Indeed, consider a Lipschitz-isomorphism $\gp:X\supseteq\supp\mu_X\to\supp\mu_Y\ssq Y$. By definition, this is an injective morphism from $\mathsf{MM}_L(X,Y)$. We need to show that $\gp$ is surjective to ensure the existence of a two-sided inverse in $\mathsf{MM}_L(Y,X)$. To this end, suppose there exists $y\in\supp\mu_Y\setminus \gp(\supp\mu_X)=:Z$. Since $\supp\mu_X$ is closed, so is its image under the homeomorphism $\gp$, and hence $Z\ssq \supp\mu_Y$ is open. As every open subset of $\supp\mu_Y$ is required to have positive measure, we obtain the contradiction
\[
  0<\mu_Y(Z)=\gp_*\mu_X(Z)=\mu_X\left(\gp^{-1}\left(\supp\mu_Y\setminus \gp(\supp\mu_X)\right)\right)=0.
\]
Hence, $\gp$ is indeed a bijection. Conversely, if $\gp$ is an isomorphism from $\mathsf{MM}_L(X,Y)$ then we get the lower Lipschitz-bound from the Lipschitz-continuity of $\gp^{-1}\in\mathsf{MM}_L(Y,X)$, thus showing that $\gp$ is also a Lipschitz-isomorphism. Analogously, the corresponding statement for mm-isomorphisms can be derived.

We will write $(X,d_X,\mu_X)\simeq (Y,d_Y,\mu_Y)$ if $X$ and $Y$ are Lipschitz-isomorphic, and 
$(X,d_X,\mu_X)\cong (Y,d_Y,\mu_Y)$ if they are mm-isomorphic. Trivially, $X\cong Y$ implies $X\simeq Y$. 

In what follows, we will always assume $\supp\mu_X=X$.
\begin{rem}
  Of course, we always have $(X,d_X,\mu_X)\cong(\supp\mu_X,d_X,\mu_X)$ by virtue of 
  $\id:X\supseteq\supp\mu_X \to\supp\mu_X$. The restriction $\supp\mu_X=X$ becomes necessary for the Einstein relation since $\dimh(\supp\mu_X)$ might be strictly smaller than $\dimh(X)$, the term appearing in the Einstein relation \eqref{eq:ER}. We will later see (Proposition \ref{prop:mmiso}) that the Einstein relation is invariant under Lipschitz-isomorphisms which provides some motivation to circumvent this restriction by considering the relation
  \[
    \dimh(\supp\mu_X)=c\dims(\supp\mu_X,A)\dimw(\supp\mu_X,M)
  \]
  instead of \eqref{eq:ER}.
\end{rem}

\subsection{Transport of structure}

Given two mm-spaces $(X,d_X,\mu_X)$ and $(Y,d_Y,\mu_Y)$ with a map $\gp:X\to Y$, where a suitable operator 
$A:L^2(X,\mu_X)\supseteq\cD(A)\to L^2(X,\mu_X)$ satisfies the Einstein relation with constant $c$ on $X$. How can we transport $A$ alongside $\gp$ to become an operator on $L^2(Y,\mu_Y)$, and which restrictions do we need to impose on $\gp$ to ensure that this transport of structure is compatible with the theory from chapter 1?

Note first that any bimeasurable bijection $\gp:(X,d_X,\mu_X)\to(Y,d_Y,\mu_Y)$ induces by precomposition an operator
\begin{align*}
  \gp_*:L^2(Y,\nu)&\to L^2(X,\mu)\\
  f(y)&\mapsto(f\circ\gp)(x)
\end{align*}
which is an isometric isomorphism because $\gp^{-1}:Y\to X$ induces its inverse and because of
\begin{equation}\label{eq:isometry}
  \left\|\gp_*f\right\|_{L^2(X,\mu_X)}^2
  =\int_X |f(\gp(x))|^2\,\Di\mu_X
  =\int_Y |f|^2\,\Di\gp_*\mu_X
  =\int_Y |f|^2\,\Di\mu_Y
  =\|f\|_{L^2(Y,\mu_Y)}^2,
\end{equation}
by the change of variables formula for Lebesgue integrals. 

Denote by $\IL(H)$ the set of all partially defined linear maps (not necessarily bounded) on a Hilbert space $H$. Given an operator $A\in\IL(L^2(X,\mu_X))$, we can now contruct an operator $\gp_{\IL} A\in\IL(L^2(Y,\mu_Y))$ by conjugating with $\gp_*$. More explicitly, we define the map
\[
  \gp_{\IL}:\IL(L^2(X,\mu))\to\IL(L^2(Y,\mu))
\]
where $(\gp_{\IL} A)f:=(\gp_*^{-1}\circ A\circ\gp_*)f$ and $\cD(\gp_{\IL} A)=\gp_*^{-1}(\cD(A))$. Note that $\gp_L$ is again a bijection with inverse given by $\gp_L^{-1}=(\gp^{-1})_L$ and that this bijection restricts to the spaces of bounded linear operators. 

It follows immediately that $\cD(\gp_{\IL} A)$ is dense iff $\cD(A)$ is, and $\gp_{\IL} A$ is self-adjoint iff $A$ is. Indeed, consider arbitrary $f,g\in\cD(\gp_{\IL} A)$ with $f=\gp_*^{-1}(\bar f)$ and $g=\gp_*^{-1}(\bar g)$, where $\bar f, \bar g\in \cD(A)$. Then, applying \eqref{eq:isometry}, we have
\[
  \left<(\gp_{\IL} A)f,g\right>_{L^2(Y,\mu_Y)}
  =\left<\gp_*^{-1}A\gp_*\gp_*^{-1}\bar f,\gp_*^{-1}\bar g\right>_{L^2(Y,\mu_Y)}
  =\left<A\bar f,\bar g\right>_{L^2(X,\mu_X)}
\]
and we can perform the same calculations for $\left<f,(\gp_{\IL} A)g\right>_{L^2(Y,\mu_Y)}$, thus establishing the claimed equivalence. It is equally straightforward to check that the resolvent sets and the eigenvalues of $A$ and $\gp_{\IL} A$ coincide: Consider $\gl\in\rho(A)$, that is, 
$(\gl-A)^{-1}$ is a bounded linear operator on $L^2(X,\mu_X)$. To show $\gl\in\rho(\gp_L A)$, we consider
\[
  (\gl-\gp_L A)^{-1}
  =\left(\gl-\gp_*^{-1}A\gp_*\right)^{-1}
  =\left(\gp_*^{-1}(\gl-A)\gp_*\right)^{-1}
  =\gp_*^{-1}(\gl-A)^{-1}\gp_*
  =\gp_L(\gl-A)^{-1}
\]
which is a bounded linear operator on $L^2(Y,\mu_Y)$. If $\gl\in\IC$ happens to be an eigenvalue of $A$ with eigenfunction $f\in L^2(X,\mu_X)$, then -- as might have been expected -- $\gp_*^{-1}f$ is an eigenfunction of 
$\gp_L A$ to the eigenvalue $\gl$ as well. This can easily be checked by calculating
\[
  (\gp_L A)\big(\gp_*^{-1}f\big)=\gp_*^{-1}Af=\gl(\gp_*^{-1}f).
\]


Moreover, $\gp_{\IL}$ respects operator semigroups: If $(T_t)_{t\geq0}$ is a strongly continuous contraction semigroup on $L^2(X,\mu)$ with generator $(-A,\cD(A))$ then $(\gp_{\IL} T_t)_{t\geq0}$ is a semigroup with the same properties on $L^2(Y,\mu_Y)$ and with generator $(-\gp_{\IL} A,\cD(\gp_{\IL} A))$. Indeed, the semigroup property is trivial to check. For contractiveness, note that for $L^2(Y,\mu_Y)\ni f=\gp_*^{-1}\bar f$ with 
$\bar f\in L^2(X,\mu_X)$
\[
  \left\|(\gp_{\IL}T_t)f\right\|_{L^2(Y,\mu_Y)}
  =\left\|\gp_*^{-1}T_t\gp_*\gp_*^{-1}\bar f\right\|_{L^2(Y,\mu_Y)}
  =\left\|T_t\bar f\right\|_{L^2(X,\mu_X)}
  \leq\|\bar f\|_{L^2(X,\mu_X)}=\|f\|_{L^2(Y,\mu_Y)}.
\]
For strong continuity, we calculate
\[
  \left\|\gp_{\IL} T_t f-\gp_{\IL} T_0 f\right\|
  =\left\|\gp_*^{-1}(T_t\gp_*f-\gp_*f)\right\|
  =\left\|T_t(\gp_*f)-(\gp_*f)\right\| \to 0
\]
for $t\searrow 0$ and arbitrary $f\in L^2(X,\mu)$, and verifying the generator works analogously. 

Note however that a bi-measurable bijection $\gp$ does not respect enough structure to ensure that 
$\gp_\IL \sqrt{A}$ generates a regular Dirichlet form if and only if $\sqrt{A}$ does -- recall that this means the density of $\cD(\sqrt{A})\cap C_c(Y)$ in both $\cD(\sqrt{A})$ and $C_c(Y)$. To this end, suppose now that $\gp:X\to Y$ is a homeomorphism between $X$ and $Y$ (since both spaces are equipped with their Borel $\gs$-algebras, such $\gp$ is automatically bi-measurable and bijective). Similar to the case of $L^2$-spaces, this induces an isometric isomorphism $\gp_*:C_0(Y)\to C_0(X), \gp_*(f)=f\circ\gp$ between algebras of continuous functions vanishing at infinity, equipped with sup-norm $\|\cdot\|_{C_0}$. This isomorphism restricts to the subalgebras of compactly supported continuous functions $C_c(X)$ resp. $C_c(Y)$. 

\begin{lem}
  With the notation just introduced, if the Dirichlet form on $L^2(X,\mu_X)$ defined by 
  \[
    \cE(f,g):=\left<\sqrt{A}f,\sqrt{A}g\right>_{L^2(X,\mu_X)}\ \text{ for } f,g\in\cD(\sqrt{A})
  \]
 is regular then so is the Dirichlet form on $L^2(Y,\mu_Y)$ defined by 
 \[
   (\gp_\IL)^*\cE(\bar f,\bar g):=\left<(\gp_\IL\sqrt{A})\bar f,(\gp_\IL\sqrt{A})\bar g\right>_{L^2(Y,\mu_Y)}\ \text{ for } \bar f,\bar g\in\gp_*^{-1}(\cD(\sqrt{A})).
 \]
\end{lem}
\begin{proof}
  We need to show that the intersection of $\cD(\gp_\IL\sqrt{A})=\gp_*^{-1}(\cD(\sqrt{A}))$ and $C_c(Y)$ is dense both in $C_c(Y)$ w.r.t $\|\cdot\|_{C_0(Y)}$ and in $\cD(\gp_\IL\sqrt{A})$ w.r.t. $((\gp_\IL)^*\cE)_1$ as introduced in definition \ref{defin:DF}.
  
  For the first part, take $C_c(Y)\ni f=\gp_*^{-1} g$ for $g\in C_c(X)$. Then, there exists a sequence $(g_n)_{n\in\IN}\ssq \cD(\sqrt{A})\cap C_c(X)$ with $\|g_n-g\|_{C_0(X)}\to 0$ as $n\to\infty$. Since $\gp_*^{-1}$ is isometric, we conlude that $f_n:=\gp_*^{-1}g_n\in \cD(\gp_\IL\sqrt{A})\cap C_c(Y)$ converges to $f$ in 
  $\|\cdot\|_{C_0(Y)}$.
  
  For the second part, we analogously take $\cD(\gp_\IL\sqrt{A})\ni f=\gp_*^{-1} g$ for $g\in\cD(\sqrt{A})$. By regularity of $\cE$, there exists again a sequence $(g_n)_{n\in\IN}\ssq\cD(\sqrt{A})\cap C_c(X)$ with $\cE_1[g_n-g]\to 0$ as $n\to\infty$. Setting $f_n:=\gp_*^{-1} g_n\in\cD(\gp_\IL\sqrt{A})\cap C_c(Y)$, we obtain
  \begin{align*}
    ((\gp_\IL)^*\cE)_1[f_n-f]
    &=\left\|\left(\gp_\IL\sqrt{A}\right)(f_n-f)\right\|_{L^2(Y,\mu_Y)}^2
        +\|f_n-f\|_{L^2(Y,\mu_Y)}^2\\
    &=\left\|\gp_*^{-1}\sqrt{A}\gp_*\gp_*^{-1}(g_n-g)\right\|_{L^2(Y,\mu_Y)}^2
        +\|\gp_*^{-1}(g_n-g)\|_{L^2(Y,\mu_Y)}^2\\
    &=\left\|\sqrt{A}(g_n-g)\right\|_{L^2(X,\mu_X)}^2+\|g_n-g\|_{L^2(X,\mu_X)}^2\\
    &=\cE_1[g_n-g]\to 0
  \end{align*}
  which concludes the proof.
\end{proof}

Putting everything together, we observe that $A$ satisfies the assumptions in \ref{cond:A} iff $\gp_\IL A$ does whenever $\gp:X\to Y$ is a homeomorphism, and then $\dims(X,A)=\dims(Y,\gp_\IL A)$. The spectral dimension is therefore stable under a very large class of transformations. As it turns out, this will not be the case for Hausdorff and walk dimension. 

\begin{prop}\label{prop:mmiso}
  Let $(X,d_X,\mu)$ and $(Y,d_Y,\nu)$ be complete separable locally compact path-connected metric measure spaces with $\supp\mu=X$ and $\supp\nu=Y$ that are Lipschitz-isomorphic by virtue of the map $\gp:X\to Y$. Suppose the Einstein relation with constant $c$ holds on $X$ with respect to an operator $(A,\cD(A))$ satisfying assumptions \ref{cond:A}. Then, the Einstein relation also holds on $Y$ with the same constant $c$ and with respect to $\gp_{\IL} A$.
\end{prop}
\begin{proof}
  As the Hausdorff dimension is invariant under bi-Lipschitz maps we obtain $\dimh(X)=\dimh(Y)$, and as observed above, $\dims(X,A)=\dims(Y,\gp_{\IL} A)$. So, it remains to show $\dimw(X,M)=\dimw(X,M^{(\gp)})$ where $M$ is a Hunt process associated to $A$ and $M^{(\gp)}$ is one associated to $\gp_{\IL} A$. 
  
  We consider the process $N_t:=\gp(M_t)$. This process is a Hunt process with values in $Y$, and possesses the semigroup
  \[
    T_t^{(N)}f
    =\PTEp{f(N_t)}{\cdot}
    =\PTEp{(f\circ\gp)(M_t)}{\gp^{-1}(\cdot)}
    =T_t[\gp_*f](\gp^{-1}(\cdot))
    =\gp_*^{-1}T_t\gp_* f
    =(\gp_{\IL} T_t)f
  \]
  where we used the notation from the discussion above. 
  
  Thus, due to theorem \ref{thm:fukushima}, the processes $N$ and $M^{(\gp)}$ coincide up to their behaviour on a polar set. It is therefore enough to determine the walk dimension for $N_t$. By the bi-Lipschitz continuity of $\gp$, we obtain 
  \[
    \gp\left(B_X\left(x,C^{-1}r\right)\right)
    \ssq B_Y\big(\gp(x),r\big) 
    \ssq \gp\big(B_X(x,Cr)\big)
  \]
  where $C>0$ is the two-sided Lipschitz constant of $\gp$. Hence, $\tau_{C^{-1}r}^{(M)}\leq \tau_r^{(N)}\leq \tau_{Cr}^{(M)}$. From this, we get for all sufficiently small $r>0$
  \[
    \frac{\log Cr}{\log r}
     \cdot\frac{\log \PTEp{\tau_{Cr}^{(M)}}{x}}{\log Cr}
    \leq \frac{\log \PTEp{\tau_r^{(N)}}{\gp(x)}}{\log r}
    \leq \frac{\log C^{-1}r}{\log r}
     \cdot\frac{\log \PTEp{\tau_{C^{-1}r}^{(M)}}{x}}{\log C^{-1}r}.
  \]
  Taking the limit for $r\searrow0$ and applying a standard squeezing argument, we obtain $\dimw(X,M)=\dimw(Y,N)$. 
\end{proof}
\begin{rem}\label{rem:ToS}
  Note that we only needed the bi-Lipschitz property for determining 
  $\dimh$ and $\dimw$, whereas we only needed $\gp$ to be a homeomorphism in order to show that $M^{(\gp)}$ and $N$ share the same semigroup. This allows us in the following sections -- given a homeomorphism $\gp:(X,d_X,\mu_X)\to(Y,d_Y)$ -- to transport the complete structure needed for the Einstein relation by 
  \begin{itemize}
    \item Endowing $(Y,\mu_Y)$ with the push-forward measure $\gp_*\mu_X$.
    \item Mapping the generator $(A,\cD(A))$ to 
    $(\gp_LA,\gp_*^{-1}\cD(A))$, thus also mapping the generated semigroup $T_t$ to $\gp_LT_t$.
    \item Sending the Hunt process $M$ to $\gp(M)$. 
  \end{itemize}
  What we did so far ensures that all these constructions are compatible with each other. 
\end{rem}

From proposition \ref{prop:mmiso} we immediately obtain the following two corollaries:
\begin{cor}
  If $(X,d_X,\mu_X)\cong(Y,d_Y,\mu_Y)$ and the Einstein relation with constant $c$ holds on $X$ w.r.t. $(A,\cD(A))$ is an operator on $L^2(X,\mu)$, then it also holds on $Y$ with the same constant w.r.t. $\gp_{\IL}A$. 
\end{cor}
\begin{cor}
  If $X\ssq\IR^n$ and $d_1$ and $d_2$ are metrics which are induced by norms, then $\id_X:(X,d_1,\mu)\to(X,d_2,\mu)$ will preserve the constant in the Einstein relation.
\end{cor}
The second corollary follows from the well-known fact that all norms on a finite-dimensional Banach space are equivalent. 



\section{H\"older regularity and graphs of functions}

A natural question arising at this point is whether the invariance of the Einstein relation of propsition \ref{prop:mmiso} can be extended to a larger class of transformations. The following family of examples shows that this is not the case:

Consider a 1-dimensional continuous $\ga$-self-similar time-homogeneous strong Markov process $(M_t)_{t\geq0}$ over a suitable probability space $(\gO,\scA,\Prob)$. Here, $\ga$-self-similar for $0<\ga\leq1$ means that the processes 
$(M_t)_{t\geq0}$ and $\left(\xi^{-\ga}M_{\xi t}\right)_{t\geq0}$ have the same distribution. Denote by
\[
 \gr(M)=\gr(M_\cdot(\go)):=\left\{(t,M_t(\go)):t\geq0 \right\}\ssq \IR^2
\]
the graph of $M$ for fixed $\go\in\gO$. 

We endow $\gr(M)$ with the metric $d_\infty$ obtained by restricting the maximum norm $|(x,y)|_\infty=\max\{|x|,|y|\}$ of the surrounding space $\IR^2$ to $\gr(M)$. By construction, $\gr(M)$ comes with the natural map $\gp=\gp_\go:\IR_{\geq0}\to\gr(M)$ sending $t$ to 
$(t,M_t)$, providing a homeomorphism between $[0,1]$ and $\gr(M)$. 
While $\gp^{-1}$ is always Lipschitz-continuous since it is the projection onto the first coordinate, $\gp$ is in general only H\"older continuous with an exponent strictly smaller than 1. In fact, $\gp_\go$ is $\gc$-H\"older continuous if and only if the trajectory $M_\cdot(\go)$ is.

As pointed out in remark \ref{rem:ToS}, we can now transfer the analytic structure of section 2.1 on the half-line $\IR_{\geq0}$ via $\gp$ to 
$\gr(M)$. More explicitly, we have the measure $\gp_*\gl^1$ on $\gr(M)$ and an operator $\gp_L\gD_{\gl^1}$ acting on $L^2(\gr(M),\gp_*\gl^1)$ that generates the Hunt process $(\gp(W_s^t))_{s\geq0}$, where 
$(W_s^t)_{s\geq0}$ is a Wiener process independent from $M$ with start in $t\geq0$.

We note the following:
\begin{itemize}
  \item From \cite{liu1998hausdorff}, we get $\dimh(\gr(M))=2-\ga$ provided $M$ satisfies the technical conditions that there exist positive constants $K,K'$ and $0<r_0<1/3$ such that 
  \begin{equation}
    p(1,x,B(x,r_0))\geq K\ \ \text{ and }\ \ 
    p(1,x,B(x,r))\leq K'(1\wedge r)
  \end{equation}
  holds for all $x\in\IR$ and every sufficiently small $r>0$. Here, $p(t,x,A)$ denotes the transition function for $M$.
  \item As discussed in the last section we have
  \[
    \frac{1}{2}=\dims\left([0,1],-\frac{1}{2}\gD_{\gl^1}\right)
    =\dims\left(\gr(B),\Phi \left(-\frac{1}{2}\gD_{\gl^1}\right)\right),
  \]
  where we once again appealed to Weyl's classical results. 
  \item It remains to evaluate the walk dimension for $\gp(W_s)$ on 
  $\gr(M)$. As will be seen in the subsequent lemmata, we have 
  $\dimw(\gr(M_\cdot(\go)),\gp(W))=\frac{2}{\ga}$ $\Prob$-almost surely.
\end{itemize}
The Einstein relation -- despite holding with constant 1 on 
$\left([0,1],-\frac{1}{2}\gD_{\gl^1},W\right)$ -- therefore changes its constant under application of $\gp$ to
\[
  c(\ga)=\ga(2-\ga),\ \ \ga\in(0,1].
\]
\begin{lem}
  With the notation just introduced, we have 
  \begin{equation}\label{eq:BMdimw}
    \dimw\left(\gr(M_\cdot(\go)),\gp_\go(W),\big(T,M_T(\go)\big)\right)
    =\frac{2}{\ga}
  \end{equation}
  $\Prob$-almost surely for each $T>0$.
\end{lem}
\begin{proof}
  For brevity, set $x=x(\go)=(T,M_T(\go))\in\gr(M_\cdot(\go))$ and chose $r>0$ small enough for $B_\infty(x,r)\ssq\IR_{\geq0}\times\IR$, where 
  $B_\infty(x,r)$ stands for the open ball of radius $r$ around $x$ with respect to $d_\infty$.
  
  We begin by introducing the random times 
  $\gT^+_r\big(M_\cdot(\go),T\big),\gT^-_r\big(M_\cdot(\go),T\big)$ to denote the time where the Markov process $M$ first resp. last exits 
  $B_\infty(x,r)$ -- in other words,
  \begin{align}\label{eq:Theta}
    \gT^+_r\big(M_\cdot(\go),T\big)&:=\inf\left\{1\geq t>T:M_t\notin B_\infty(x,r)\right\}\notag\\
    \gT^-_r\big(M_\cdot(\go),T\big)&:=\sup\left\{0\leq t<T:M_t\notin B_\infty(x,r)\right\}.
  \end{align}
  Note first that by the Markov property of Brownian motion as well as by its time symmetry, these random variables are independent and the shifted random variables $\gT^+_r\big(B_\cdot(\go),T\big)-T,T-\gT^-_r\big(B_\cdot(\go),T\big)$ share the same distribution.
  
  By the standard result for the expectation of two-sided exit times for Brownian motion, we now obtain 
  \[
    \PTEp{\tau_{B_\infty(x,r)}^{(\gp(\bar B))}}{x}=-\big(T^+_r(\go)-T\big)\big(T^-_r(\go)-T\big)
  \]
  and consequentially
  \[
    \frac{\log  \PTEp{\tau_{B_\infty(x,r)}^{(\gp(\bar B))}}{x}}{\log r}
    =\frac{\log \big(\gT^+_r\big(B_\cdot(\go),T\big)-T\big)}{\log r}
     +\frac{\log \big(T-\gT^-_r\big(B_\cdot(\go),T\big))\big)}{\log r}.
  \]
  Therefore, it will suffice to show that $\Prob$-almost surely, 
  \begin{equation}\label{eq:limtwo}
    \lim_{r\searrow 0}\frac{\log \big(\gT^+_r\big(B_\cdot(\go),T\big)-T\big)}{\log r}
    =\lim_{r\searrow 0}\frac{\log \big(T-\gT^-_r\big(B_\cdot(\go),T\big)\big)}{\log r}=2,
  \end{equation}
  where it is enough to prove that one of the limits exist and equals 2. To this end, we consider more generally for a continuous function $f:[0,1]\to \IR$ the functionals
  \begin{align*}
    w:f(T)\mapsto &\liminf_{r\searrow0}\frac{\log\big(\gT^+_r(f,T)-T\big)}{\log r}\\
    W:f(T)\mapsto &\limsup_{r\searrow0}\frac{\log\big(\gT^+_r(f,T)-T\big)}{\log r}
  \end{align*}
  where $\gT^\pm_r(f,T)$ is defined analogously to \eqref{eq:Theta}. Suppose now that $f$ is $\ga$-H\"{o}lder 
  continuous ($0<\ga\leq 1$) in a given point $T\in[0,1]$, that is, there exists a constant $C>0$ and a 
  $\gep$-neighbourhood of $T$ such that for all $s$ inside this neighbourhood,
  \[
    |f(T)-f(s)|\leq C|T-s|^\ga
  \]
  is satisfied. This yields $\gT^+_r(f,T)\geq (r/C)^{1/\ga}$ for $r<\gep$ and therefore $(Wf)(T)\leq \frac{1}{\ga}$. Conversely, suppose that $f$ is not $\gb$-H\"{o}lder continuous ($\ga<\gb\leq 1$) in $T$. In particular, there exists a sequence $s_n=T+r_n\to T$ fulfilling the estimate 
  \[
    |f(T)-f(s_n)|>|T-s_n|^\gb=r_n^\gb,
  \]
  from which we deduce $\gT^+_{r_n}(f,T)\leq r_n^{1/\gb}$ and thus $(wf)(T)\geq\frac{1}{\gb}$. In particular, 
  $(Wf)(T)=(wf)(T)=\frac{1}{c}$ if $f$ is $\ga$-H\"{o}lder continuous in $T$ for all $\ga<c$ but not $\gb$-H\"{o}lder continuous for any $\gb>c$. 
  
  Having established this claim, the limit in equation \eqref{eq:limtwo} is a direct consequence of Paley-Zygmunds regularity theorem for paths of Brownian motion. Moreover, this also shows that the limit does not depend on $T\in(0,1)$. 
\end{proof}

